% !TeX program = pdflatex
\documentclass[12pt,letterpaper]{article}
\usepackage[utf8x]{inputenc}
\usepackage{ucs}
\usepackage[spanish,mexico]{babel}
\usepackage{amsmath}
\usepackage{amsfonts}
\usepackage{amssymb}
\usepackage{makeidx}
\usepackage{xcolor}
\usepackage{graphicx}
\usepackage[useregional]{datetime2}
\usepackage{fancyhdr}
% % % %Geometry
\usepackage{geometry}%\usepackage[showframe]{geometry}
%\usepackage{layout}
\setlength{\voffset}{-0.7in}
\setlength{\headsep}{10pt}
\setlength{\textheight}{10.5in}
\usepackage{wasysym} %emoticons :)
\usepackage[oldstyle]{kpfonts}
%\usepackage[T1]{fontenc}
\newcommand{\fej}{\relax\hfill\ifmmode{\lower.5ex\hbox{{\textcolor{blue}{\LARGE\smiley al 15pt}}}}\else\lower.5ex\hbox{{\textcolor{blue}{\LARGE \smiley}}}}  % Smiley emoticon :)
\author{\textsc{Manuel López Mateos}}
%%% Octubre 7, 2018
% % % % % % % Para usar título, autor y fecha por separado.
\makeatletter
\let\newtitle\@title
\let\elautor\@author
\let\newdate\@date
\makeatother
%
%
% % % % Enviroments
\newenvironment{definition}[1][Definición.]{\begin{trivlist}
		\item[\hskip \labelsep {\bfseries #1}]}{\end{trivlist}}
% % % % % % % % % % % %
%
% % % % % % % % Headers
\pagestyle{fancy}
\fancyhf{}
\rhead{\color{olive}\hfill \DTMnow}
\lhead{\color{olive}\elautor}
\cfoot{\thepage}
\renewcommand{\headrule}{\color{olive}\hrule}
%\rfoot{}
% % % % % % %
%\title{Sumas de Riemann}
%\author{Manuel López Mateos}
\linespread{1.08}         % Palatino needs more leading (space between lines)
\begin{document}
\noindent Si la función real de variable real $f(x)=x^2+1$ está definida en el intervalo cerrado $[-1,1]$, encuentra la suma de Riemann correspondiente a una partición del intervalo que lo divide en cuatro subintervalos iguales.

La partición que divide a $[-1,1]$ en cuatro subintervalos iguales es $$\pi=\{x_0=-1, x_1=-0.5, x_2=0, x_3=0.5, x_4=1\}.$$

La suma de Riemann se define como 
$$\mathcal S=\sum_{i=1}^4f(x_i)(x_i-x_{i-1}).$$

Es decir, se consideran los rectángulos cuya base es cada subintervalo y su altura el valor de $f$ en el lado derecho del subintervalo.

En este caso los subintervalos tienen la misma longitud, que es $0.5$ (¿por qué?), la longitud de la base de los rectángulos. La altura de cada rectángulo es $f(x_i)$, para $i=1$, $2$, $3$, $4$.

Así,
\begin{align*}
\mathcal S&=\left(f(-0.5)+f(0)+f(0.5)+f(1)\right)0.5\\
&=\left((-0.5)^2+1+0^2+1+0.5^2+1+1^2+1\right)0.5\\
&=(1.25+1+1.25+2)0.5\\
&=(5.5)(0.5)\\
&=2.75
\end{align*}

\fej
\end{document}