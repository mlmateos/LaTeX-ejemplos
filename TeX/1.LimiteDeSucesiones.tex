% !TeX TS-program = lualatex

\documentclass[12pt,letterpaper]{article}
\usepackage{fontspec}
\usepackage{amsmath}% if desired
\usepackage{unicode-math}
\setmathfont[math-style=upright]{Neo Euler}
\setmainfont{Charis SIL}[%
Numbers = {OldStyle, Proportional},
Ligatures = TeX,
Contextuals = WordFinal,	
ItalicFont = Charis SIL Italic,
BoldFont = Charis SIL Bold,
BoldItalicFont = Charis SIL Bold Italic,
]

\setsansfont{TeX Gyre Adventor}[Scale=MatchLowercase]
\setmonofont{Courier New}
\linespread{1.05}

\usepackage[spanish,mexico]{babel}
\usepackage{xcolor}
\usepackage{graphicx}
\usepackage[useregional]{datetime2}
\usepackage{fancyhdr}
\usepackage{csquotes}

% % % %Geometry
\usepackage{geometry}
\setlength{\voffset}{-0.7in}
\setlength{\headsep}{10pt}
\setlength{\textheight}{10.5in}
\usepackage{wasysym} %emoticons :)
\newcommand{\fej}{\relax\hfill\ifmmode{\lower.5ex\hbox{{\textcolor{blue}{\LARGE\smiley al 15pt}}}}\else\lower.5ex\hbox{{\textcolor{blue}{\LARGE \smiley}}}}  % Smiley emoticon :)
\author{\textsc{Manuel López Mateos}}
% % % % % % % Para usar título, autor y fecha por separado.
\makeatletter
\let\newtitle\@title
\let\elautor\@author
\let\newdate\@date
\makeatother
%
%
% % % % Enviroments
\newenvironment{definition}[1][Definición.]{\begin{trivlist}
\item[\hskip \labelsep {\bfseries #1}]}{\end{trivlist}}
% % % % % % % % % % % %
%
% % % % % % % % Headers
\pagestyle{fancy}
\fancyhf{}
\rhead{\color{olive}\hfill \DTMnow}
\lhead{\color{olive}\elautor}
\cfoot{\thepage}
\renewcommand{\headrule}{\color{olive}\hrule}
% % % % %
\begin{document}
\bigskip 

\noindent ¿Cuál es el \emph{límite} de una sucesión?

\medskip
\noindent Primero veamos qué es una \emph{sucesión}.
\begin{definition}Una \emph{\color{purple}sucesión de números reales} es una función cuyo dominio es el conjunto $\mathbb N$ de los números naturales, y su contradominio es $\mathbb R$, el conjunto de los números reales. 
\end{definition}	
En lugar de usar la notación de funciones $S\colon \mathbb N \to \mathbb R$ y regla de correspondencia $S(n)$, simplemente llamamos $S_n$ a la imagen de $n$ bajo $S$ y le llamamos el \emph{\color{purple} término $n$-ésimo} de la sucesión.

Podemos \emph{listar} los términos de la sucesión:
$$S=\{S_1,S_2,S_3,\ldots\}.$$
	
Si $S_n=2+\dfrac{(-1)^n}{n}$, la sucesión es
$$S=\left\{1, 2+\frac{1}{2}, 2-\frac{1}{3}, 2+\frac{1}{4},\ldots\right\}.$$
	
En este caso, los términos de la sucesión \emph{brincan} alrededor del número $2$ conforme se toman valores más grandes de $n$. $S_1=1$, después pasa a $S_2=5/2$, \emph{rebota} a $S_3=5/3$, pasa al otro lado del $2$ con $S_4=5/4$, regresa al lado izquierdo del $2$ con $S_5=9/5$, y así sucesivamente. Es decir, conforme crece el valor de $n$, a partir de \emph{algún} término, el valor correspondiente $S_n$ está \emph{más y más cerca} del $2$. Decimos entonces que el límite de $S_n$ es $2$ cuando $n$ \emph{\color{purple}tiende a} infinito.

\begin{quotation}
	\noindent La frase \enquote{\emph{más y más cerca}} significa que, a partir de \emph{algún} término de la sucesión, los que le siguen distan del número $2$ en menos que cualquier distancia \emph{prefijada}.
\end{quotation}


\fej
\end{document}