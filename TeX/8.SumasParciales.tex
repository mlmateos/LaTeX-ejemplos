% !TeX program = pdflatex
\documentclass[12pt,letterpaper]{article}
\usepackage[utf8x]{inputenc}
\usepackage{ucs}
\usepackage[spanish,mexico]{babel}
\usepackage{amsmath}
\usepackage{amsfonts}
\usepackage{amssymb}
\usepackage{makeidx}
\usepackage{xcolor}
\usepackage{graphicx}
\usepackage[useregional]{datetime2}
\usepackage{fancyhdr}
% % % %Geometry
\usepackage{geometry}%\usepackage[showframe]{geometry}
%\usepackage{layout}
\setlength{\voffset}{-0.7in}
\setlength{\headsep}{10pt}
\setlength{\textheight}{10.5in}
\usepackage{wasysym} %emoticons :)
\usepackage[oldstyle]{kpfonts}
%\usepackage[T1]{fontenc}
\newcommand{\fej}{\relax\hfill\ifmmode{\lower.5ex\hbox{{\textcolor{blue}{\LARGE\smiley al 15pt}}}}\else\lower.5ex\hbox{{\textcolor{blue}{\LARGE \smiley}}}}  % Smiley emoticon :)
\author{\textsc{Manuel López Mateos}}
%%% Diciembre 10,2016
% % % % % % % Para usar título, autor y fecha por separado.
\makeatletter
\let\newtitle\@title
\let\elautor\@author
\let\newdate\@date
\makeatother
%
%
% % % % Enviroments
\newenvironment{definition}[1][Definición.]{\begin{trivlist}
		\item[\hskip \labelsep {\bfseries #1}]}{\end{trivlist}}
% % % % % % % % % % % %
%
% % % % % % % % Headers
\pagestyle{fancy}
\fancyhf{}
\rhead{\color{olive}\hfill \DTMnow}
\lhead{\color{olive}\elautor}
\cfoot{\thepage}
\renewcommand{\headrule}{\color{olive}\hrule}
%\rfoot{}
% % % % % % %
%\title{Sumas parciales}
%\author{Manuel López Mateos}
\linespread{1.08}         % Palatino needs more leading (space between lines)
\begin{document}
Fórmula para las sumas parciales de 
$\displaystyle\sum_{k=2}(2k-7)$.\qquad (OJO: $k=2$)

\bigskip
Sabemos que  
\begin{align*}
\sum_{k=1}^n(2k-7)&=\big(2(1)-7\big)+\big(2(2)-7\big)+\cdots+\big(2(n)-7\big)\\
&=2(1+2+\cdots+n)-7n\\
&=2\left(\frac{n(n+1)}{2}\right)-7n\\
&=n(n+1)-7n\\
&=n^2+n-7n\\
&=n^2-6n.
\end{align*}

También sabemos que
\begin{align*}
\sum_{k=1}^n(2k-7)&=\big(2(1)-7\big)+\sum_{k=2}^n(2k-7)\\
&=-5+\sum_{k=2}^n(2k-7)
\end{align*}
de donde
\begin{align*}
\sum_{k=2}^n(2k-7)&=\sum_{k=1}^n(2k-7)-(-5)\\
&=\sum_{k=1}^n(2k-7)+5\\
&=n^2-6n+5
\end{align*}
Es decir
$$\sum_{k=2}^n(2k-7)=n^2-6n+5.$$

Para $n=9$ tenemos que $\sum_{k=2}^n(2k-7)=9^2-6(9)+5=32$.

\fej
\end{document}