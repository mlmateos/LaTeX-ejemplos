% !TeX program = pdflatex
\documentclass[12pt,letterpaper]{article}
\usepackage[utf8x]{inputenc}
\usepackage{ucs}
\usepackage[spanish,mexico]{babel}
\usepackage{amsmath}
\usepackage{amsfonts}
\usepackage{amssymb}
\usepackage{makeidx}
\usepackage{xcolor}
\usepackage{graphicx}
\usepackage[useregional]{datetime2}
\usepackage{fancyhdr}
% % % %Geometry
\usepackage{geometry}%\usepackage[showframe]{geometry}
%\usepackage{layout}
\setlength{\voffset}{-0.7in}
\setlength{\headsep}{10pt}
\setlength{\textheight}{10.5in}
\usepackage{wasysym} %emoticons :)
\usepackage[oldstyle]{kpfonts}
%\usepackage[T1]{fontenc}
\newcommand{\fej}{\relax\hfill\ifmmode{\lower.5ex\hbox{{\textcolor{blue}{\LARGE\smiley al 15pt}}}}\else\lower.5ex\hbox{{\textcolor{blue}{\LARGE \smiley}}}}  % Smiley emoticon :)
\author{\textsc{Manuel López Mateos}}
%%%% Fecha: 7 de octubre de 2018
% % % % % % % Para usar título, autor y fecha por separado.
\makeatletter
\let\newtitle\@title
\let\elautor\@author
\let\newdate\@date
\makeatother
%
%
% % % % Enviroments
\newenvironment{definition}[1][Definición.]{\begin{trivlist}
		\item[\hskip \labelsep {\bfseries #1}]}{\end{trivlist}}
% % % % % % % % % % % %
%
% % % % % % % % Headers
\pagestyle{fancy}
\fancyhf{}
\rhead{\color{olive}\hfill \DTMnow}
\lhead{\color{olive}\elautor}
\cfoot{\thepage}
\renewcommand{\headrule}{\color{olive}\hrule}
%\rfoot{}
% % % % % % %
%\title{Integración por partes}
%\author{Manuel López Mateos}
\linespread{1.08}         % Palatino needs more leading (space between lines)
\begin{document}

Obtener la integral $\displaystyle\int\!\frac{xe^{-x}}{(x-1)^2}\,dx$.

\smallskip
Para integrar por partes debemos colocar la integral en la forma $\displaystyle\int\!\! u\,dv$.

%\smallskip
Hacemos $u=xe^{-x}$ y $dv=\dfrac{dx}{(x-1)^2}$.

%\smallskip
La fórmula para integrar por partes dice que $\displaystyle\int\!\!u\,dv=uv-\int\!\!v\,du$.

Calculamos $du$ y $v$.
\begin{align*}
u&=xe^{x},\\
\noalign{\noindent\text{por lo tanto}}
du&=e^{-x}(1-x)\,dx.\\
\noalign{\text{Ahora $v$, como}}
dv&=(x-1)^{-2}\,dx,\\
v&=\int\!\frac{1}{(x-1)^2}\,dx.
\end{align*}
Para obtener esta última integral substituimos $s=x-1$, luego $ds=dx$,
$$\int\!\frac{1}{(x-1)^2}\,dx=\int\!\frac{1}{s^2}\,ds=\frac{1}{-s}=\frac{1}{-(x-1)}=\frac{1}{1-x}.$$

Ahora aplicamos la fórmula de la integración por partes,
\begin{align*}
\int\!\frac{xe^{-x}}{(x-1)^2}\,dx&=(xe^{-x})\left(\frac{1}{1-x}\right)-\int\!\frac{1}{1-x}e^{-x}(1-x)\,dx\\
\noalign{\smallskip}
&=\frac{xe^{-x}}{1-x}-\int\!e^{-x}\,dx\\
\noalign{\smallskip}
&=\frac{xe^{-x}}{1-x}+e^{-x}\\
\noalign{\smallskip}
&=-\frac{e^{-x}}{x-1}+\text{\textcolor{gray}{constante}}.
\end{align*}

\fej
\end{document}