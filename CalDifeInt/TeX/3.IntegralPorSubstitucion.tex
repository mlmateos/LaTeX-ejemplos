% !TeX program = pdflatex
\documentclass[12pt,letterpaper]{article}
\usepackage[utf8x]{inputenc}
\usepackage{ucs}
\usepackage[spanish,mexico]{babel}
\usepackage{amsmath}
\usepackage{amsfonts}
\usepackage{amssymb}
\usepackage{makeidx}
\usepackage{xcolor}
\usepackage{graphicx}
\usepackage[useregional]{datetime2}
\usepackage{fancyhdr}
% % % %Geometry
\usepackage{geometry}%\usepackage[showframe]{geometry}
%\usepackage{layout}
\setlength{\voffset}{-0.7in}
\setlength{\headsep}{10pt}
\setlength{\textheight}{10.5in}
\usepackage{wasysym} %emoticons :)
\usepackage[oldstyle]{kpfonts}
%\usepackage[T1]{fontenc}
\newcommand{\fej}{\relax\hfill\ifmmode{\lower.5ex\hbox{{\textcolor{blue}{\LARGE\smiley al 15pt}}}}\else\lower.5ex\hbox{{\textcolor{blue}{\LARGE \smiley}}}}  % Smiley emoticon :)
\author{\textsc{Manuel López Mateos}}
% % % % % % % Para usar título, autor y fecha por separado.
\makeatletter
\let\newtitle\@title
\let\elautor\@author
\let\newdate\@date
\makeatother
%
%
% % % % Enviroments
\newenvironment{definition}[1][Definición.]{\begin{trivlist}
		\item[\hskip \labelsep {\bfseries #1}]}{\end{trivlist}}
% % % % % % % % % % % %
%
% % % % % % % % Headers
\pagestyle{fancy}
\fancyhf{}
\rhead{\color{olive}\hfill \DTMnow}
\lhead{\color{olive}\elautor}
\cfoot{\thepage}
\renewcommand{\headrule}{\color{olive}\hrule}
%\rfoot{}
% % % % % % %
%\title{Sumas parciales}
%\author{Manuel López Mateos}
\linespread{1.08}         % Palatino needs more leading (space between lines)
\begin{document}
Obtener la integral $\displaystyle\int \frac{5x-6}{2x^2-4x+3}dx$.

\bigskip
Tratemos de colocarla en la forma $\displaystyle\int \frac{du}{u}$. 

\smallskip
Si $u=2x^2-4x+3$, entonces $du=(4x-4)dx$.

\medskip
Reescribimos el numerador como
\begin{align*}
5x-6&=5x-5-1\\
&=5(x-1)-1.
\end{align*}

El integrando se reescribe como
\begin{align*}
\frac{5x-6}{2x^2-4x+3}&=\frac{5(x-1)}{2x^2-4x+3}-\frac{1}{2x^2-4x+3},\\
\noalign{\bigskip\noindent\text multiplicando por $4$ y dividiendo entre $4$ la primera fracción,\smallskip}
&=\frac{5(4x-4)}{4(2x^2-4x+3)}-\frac{1}{2x^2-4x+3}.
\end{align*}

La integral queda entonces como
\begin{align*}
\int \frac{5x-6}{2x^2-4x+3}dx&=\int\left(\frac{5(4x-4)}{4(2x^2-4x+3)}-\frac{1}{2x^2-4x+3}\right)dx\\
\noalign{\bigskip}
&=\frac{5}{4}\int\frac{4x-4}{2x^2-4x+3}dx-\int\frac{1}{2x^2-4x+3}dx.
\end{align*}

La primera ya está en la forma $\displaystyle\int \frac{du}{u}$ y la segunda se resuelve completando cuadrados.

\fej


\end{document}