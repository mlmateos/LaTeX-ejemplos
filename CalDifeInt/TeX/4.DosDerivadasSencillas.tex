% !TeX TS-program = pdflatex

\documentclass[12pt,letterpaper]{article}
\usepackage[utf8x]{inputenc}
\usepackage{ucs}
\usepackage[spanish,mexico]{babel}
\usepackage{amsmath}
\usepackage{amsfonts}
\usepackage{amssymb}
\usepackage{makeidx}
\usepackage{xcolor}
\usepackage[useregional]{datetime2}
\usepackage{fancyhdr}
% % % %Geometry
\usepackage{geometry}%\usepackage[showframe]{geometry}
%\usepackage{layout}
\setlength{\voffset}{-0.7in}
\setlength{\headsep}{10pt}
\setlength{\textheight}{10.5in}
\usepackage{wasysym} %emoticons :)
\usepackage[oldstyle]{kpfonts}
%\usepackage[T1]{fontenc}
\newcommand{\fej}{\relax\hfill\ifmmode{\lower.5ex\hbox{{\textcolor{blue}{\LARGE\smiley}}}}\else\lower.5ex\hbox{{\textcolor{blue}{\LARGE \smiley}}}}  % Smiley emoticon :)
\author{Manuel López Mateos}
% % % % % % % Para usar título, autor y fecha por separado.
\makeatletter
\let\newtitle\@title
\let\elautor\@author
\let\newdate\@date
\makeatother
%
% % % % % % % % Headers
\pagestyle{fancy}
\fancyhf{}
\rhead{\color{olive}\hfill \DTMnow}
\lhead{\color{olive}\elautor}
\cfoot{\thepage}
\renewcommand{\headrule}{\color{olive}\hrule}
%\rfoot{}
% % % % % % %
\begin{document}\ 
	\bigskip
	
\noindent ¿Quién no se  complica las cosas?
Obtener la derivada de $y=\dfrac{4}{x^2-12}$.

\medskip
Vemos una fracción y pensamos en usar la fórmula para la derivada de un cociente: 
$$\left(\frac{f}{g}\right)'(x)=\frac{g(x)f'(x)-f(x)g'(x)}{\left[g(x)\right]^2}.$$

\medskip
Esta fórmula tiene hasta una tonadita para recordarla: 

\begin{quotation}
\noindent\emph{La de abajo por la derivada de la de arriba menos la de arriba por la derivada de la de abajo, sobre la de abajo al cuadrado}.
\end{quotation}

Así, 
\begin{align*}
y'&=\frac{(x^2-12)(0)-4(2x)}{(x^2-12)^2}\\
&=\frac{-8x}{(x^2-12)^2}.
\end{align*}
%\smallskip

Pero, sin complicarnos, notemos que $y=\dfrac{4}{x^2-12}=4(x^2-12)^{-1}$.

Luego
\begin{align*}
y'&=(-1)4(x^2-12)^{-1-1}(2x)\\
&=-4(x^2-12)^{-2}(2x)\\
&=-8x(x^2-12)^{-2}\\
&=\frac{-8x}{(x^2-12)^2}.
\end{align*}

Naturalmente, al derivar $(x^2-12)^{-2}$, aplicamos la \emph{regla de la cadena} al \emph{multiplicar por la derivada de lo de adentro}, en este caso $2x$. 

\fej
\end{document}