% !TeX program = pdflatex
\documentclass[12pt,letterpaper]{article}
\usepackage[utf8x]{inputenc}
\usepackage{ucs}
\usepackage[spanish,mexico]{babel}
\usepackage{amsmath}
\usepackage{amsfonts}
\usepackage{amssymb}
\usepackage{makeidx}
\usepackage{xcolor}
\usepackage{graphicx}
\usepackage[useregional]{datetime2}
\usepackage{fancyhdr}
% % % %Geometry
\usepackage{geometry}%\usepackage[showframe]{geometry}
%\usepackage{layout}
\setlength{\voffset}{-0.7in}
\setlength{\headsep}{10pt}
\setlength{\textheight}{10.5in}
\usepackage{wasysym} %emoticons :)
\usepackage[oldstyle]{kpfonts}
%\usepackage[T1]{fontenc}
\newcommand{\fej}{\relax\hfill\ifmmode{\lower.5ex\hbox{{\textcolor{blue}{\LARGE\smiley al 15pt}}}}\else\lower.5ex\hbox{{\textcolor{blue}{\LARGE \smiley}}}}  % Smiley emoticon :)
\author{\textsc{Manuel López Mateos}}
% % % % % % % Para usar título, autor y fecha por separado.
\makeatletter
\let\newtitle\@title
\let\elautor\@author
\let\newdate\@date
\makeatother
%
%
% % % % Enviroments
\newenvironment{definition}[1][Definición.]{\begin{trivlist}
		\item[\hskip \labelsep {\bfseries #1}]}{\end{trivlist}}
% % % % % % % % % % % %
%
% % % % % % % % Headers
\pagestyle{fancy}
\fancyhf{}
\rhead{\color{olive}\hfill \DTMnow}
\lhead{\color{olive}\elautor}
\cfoot{\thepage}
\renewcommand{\headrule}{\color{olive}\hrule}
%\rfoot{}
% % % % % % %
%\title{Función piso}
\author{Manuel López Mateos}
\linespread{1.08}         % Palatino needs more leading (space between lines)
\begin{document}
%\maketitle
\noindent La {\color{purple}\textbf{función piso}} está definida en el conjunto de los números reales $\mathbb R$. Asocia a cada $x\in \mathbb{R}$ el máximo entero menor o igual que $x$. Se denota con
$$\text{piso}(x)=\lfloor x\rfloor=\text{máximo entero menor o igual a $x$}.$$ 
Por ejemplo $\lfloor 1.83\rfloor=1$, $\lfloor 97\rfloor=97$, $\lfloor\pi\rfloor=3$, $\lfloor-\pi\rfloor=-4$.

\bigskip
\noindent \textbf{Encuentra los elementos $\max A$ y $\min A$ para }
$$A=\left\{x^2\,\left\lfloor|x-1|+2\right\rfloor+\lfloor x\rfloor,\ \text{tal que}\ x\in(0,2]\right\}.$$

Veamos cómo se comporta el número $x^2\,\left\lfloor|x-1|+2\right\rfloor+\lfloor x\rfloor$ para $x\in(0,1)$, $x\in[1,2)$ y $x=2$.

Si $x$ está en el intervalo abierto $(0,1)$ su distancia al $1$ siempre será menor que $1$ pero mayor que $0$ (pues $1\notin(0,1)$), es decir $0<|x-1|<1$. Sumando $2$ en cada miembro de la desigualdad obtenemos $2<|x-1|+2<3$, luego $\left\lfloor|x-1|+2\right\rfloor=2$ y $x^2\,\left\lfloor|x-1|+2\right\rfloor=2x^2$. Como $x\in(0,1)$, $\lfloor x\rfloor=0$. Tenemos entonces que $x^2\,\left\lfloor|x-1|+2\right\rfloor+\lfloor x\rfloor=2x^2$.

Si $x$ está en el intervalo semiabierto $[1,2)$ su distancia al $1$ será menor que $1$ y mayor o igual que $0$ (pues $|1-1|=0$), es decir $0\leq|x-1|<1$, Sumando $2$ en cada miembro de la desigualdad obtenemos $2\leq|x-1|+2<3$, luego $\left\lfloor|x-1|+2\right\rfloor=2$ y $x^2\,\left\lfloor|x-1|+2\right\rfloor=2x^2$. Como $x\in[1,2)$, $\lfloor x\rfloor=1$. Tenemos entonces que $x^2\,\left\lfloor|x-1|+2\right\rfloor+\lfloor x\rfloor=2x^2+1$.

Finalmente, si $x=2$ entonces $\left\lfloor|x-1|+2\right\rfloor=3$ y 
$x^2\,\left\lfloor|x-1|+2\right\rfloor=3x^2$, $\lfloor x\rfloor=2$ y 
$x^2\,\left\lfloor|x-1|+2\right\rfloor+\lfloor x\rfloor=3x^2+2=14$.

Es decir,
  \[
    x^2\,\left\lfloor|x-1|+2\right\rfloor+\lfloor x\rfloor = \left\{\begin{array}{ll}
        2x^2, & \text{para } 0< x< 1\\
        2x^2+1, & \text{para } 1\leq x< 2\\
        14, & \text{para } x=2
        \end{array}\right.
  \]
  
  Por inspección vemos que el valor máximo que puede alcanzar en $A$ el número $x^2\,\left\lfloor|x-1|+2\right\rfloor+\lfloor x\rfloor$  es $14$. Así, $\max A=14$.

El mínimo de los valores de $x^2\,\left\lfloor|x-1|+2\right\rfloor+\lfloor x\rfloor$  es $0$ cuando $x=0$, pero $0$ no es un elemento de $A$ luego el $\min A$ no existe. No hay un elemento de $A$ que sea menor o igual que cualquier otro elemento de $A$ (¿Lo puedes demostrar?).

\fej
  
\end{document}