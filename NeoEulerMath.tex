% !TeX TS-program = lualatex
%%%%%%%% SOURCE %%%%%%
% User: Davislor
% Euler in Modern Toolchains
% April 11, 2018
% https://tex.stackexchange.com/a/425887/18280
%%%%%%%%
\documentclass{article}
\pagestyle{empty}

\usepackage{amsmath}
\DeclareMathOperator\Res{Res}
\newcommand*\diff{\mathop{}\!\mathup{d}}

\usepackage{amsthm}
\newtheorem{theorem}{Theorem}

\usepackage{unicode-math}
\usepackage{hyperref}

%%%
% Set up you text and math fonts
%%%

\unimathsetup{math-style=upright}
\setmainfont{CMU Concrete}
\defaultfontfeatures{Scale=MatchLowercase}
\setmathfont{Libertinus Math}
\setmathfont[range={"0000-"0001,"0020-"007E,
	"00A0,"00A7-"00A8,"00AC,"00AF,"00B1,"00B4-"00B5,"00B7,
	"00D7,"00F7,
	"0131,
	"0237,"02C6-"02C7,"02D8-"02DA,"02DC,
	"0300-"030C,"030F,"0311,"0323-"0325,"032E-"0332,"0338,
	"0391-"0393,"0395-"03A1,"03A3-"03A8,"03B1-"03BB,
	"03BD-"03C1,"03C3-"03C9,"03D1,"03D5-"03D6,"03F5,
	"2016,"2018-"2019,"2021,"2026-"202C,"2032-"2037,"2044,
	"2057,"20D6-"20D7,"20DB-"20DD,"20E1,"20EE-"20EF,
	"210B-"210C,"210E-"2113,"2118,"211B-"211C,"2126-"2128,
	"212C-"212D,"2130-"2131,"2133,"2135,"2190-"2199,
	"21A4,"21A6,"21A9-"21AA,"21BC-"21CC,"21D0-"21D5,
	"2200,"2202-"2209,"220B-"220C,"220F-"2213,"2215-"221E,
	"2223,"2225,"2227-"222E,"2234-"2235,"2237-"223D,
	"2240-"224C,"2260-"2269,"226E-"2279,"2282-"228B,"228E,
	"2291-"2292,"2295-"2299,"22A2-"22A5,"22C0-"22C5,
	"22DC-"22DD,"22EF,"22F0-"22F1,
	"2308-"230B,"2320-"2321,"2329-"232A,"239B-"23AE,
	"23DC-"23DF,
	"27E8-"27E9,"27F5-"27FE,"2A0C,"2B1A,
	"1D400-"1D433,"1D49C,"1D49E-"1D49F,"1D4A2,"1D4A5-"1D4A6,
	"1D4A9-"1D4AC,"1D4AE-"1D4B5,"1D4D0-"1D4E9,"1D504-"1D505,
	"1D507-"1D50A,"1D50D-"1D514,"1D516-"1D51C,"1D51E-"1D537,
	"1D56C-"1D59F,"1D6A8-"1D6B8,"1D6BA-"1D6D2,"1D6D4-"1D6DD,
	"1D6DF,"1D6E1,"1D7CE-"1D7D7
}]{Neo Euler}
\setmathfont[range=up/{greek,Greek}, script-features={}, sscript-features={}
]{Neo Euler}
\setmathfont[range=up/{latin,Latin,num}, script-features={}, sscript-features={}
]{Neo Euler}

\begin{document}
	
	\begin{theorem}[Residue theorem]
		Let $f$ be analytic in the region $G$ except for the isolated
		singularities $a_1,a_2,\dots,a_m$. If $\gamma$ is a closed
		rectifiable curve in $G$ which does not pass through any of the
		points $a_k$ and if $\gamma\approx 0$ in $G$, then
		\[
		\frac{1}{2\symup{\pi i}} \int\limits_\gamma f\Bigl(x^{\mathbf{N}\in\mathbb{C}^{N\times 10}}\Bigr)
		= \sum_{k=1}^m n(\gamma;a_k)\Res(f;a_k)\,.
		\]
	\end{theorem}
	
	\begin{theorem}[Maximum modulus]
		Let $G$ be a bounded open set in $\BbbC$ and suppose that $f$ is a
		continuous function on $G^-$ which is analytic in $G$. Then
		\[
		\max\{\, |f(z)|:z\in G^- \,\} = \max\{\, |f(z)|:z\in \partial G \,\}\,.
		\]
	\end{theorem}
	
	First some large operators both in text:
	$\iiint\limits_{Q}f(x,y,z) \diff x \diff y \diff z$
	and
	$\prod_{\gamma\in\Gamma_{\bar{C}}}\partial(\tilde{X}_\gamma)$;\
	and also on display
	\[
	\iiiint\limits_{Q}f(w,x,y,z) \diff w \diff x \diff y \diff z
	\leq
	\oint_{\partial Q} f'\Biggl(\max\Biggl\{
	\frac{\Vert w\Vert}{\vert w^2+x^2\vert};
	\frac{\Vert z\Vert}{\vert y^2+z^2\vert};
	\frac{\Vert w\oplus z\Vert}{\vert x\oplus y\vert}
	\Biggr\}\Biggr)\,.
	\]
	
	\vfill
	
	\noindent SOURCE

	 User: Davislor
	 
	 \textbf{Euler in Modern Toolchains}
	 
	 April 11, 2018
	 
\url{https://tex.stackexchange.com/a/425887/18280}

Last visited: \today
\end{document}