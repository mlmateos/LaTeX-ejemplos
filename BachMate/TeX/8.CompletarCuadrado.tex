% !TeX TS-program = lualatex

\documentclass[12pt,letterpaper]{article}
%%% LuaLaTex
\usepackage{fontspec}
\usepackage{amsmath}% if desired
\usepackage{unicode-math}
\renewcommand{\boldsymbol}{\symbf}

\setmainfont{TeX Gyre Pagella}[
Numbers	=	{OldStyle, Proportional},
Ligatures	=	TeX,
%Script=Arabic
%Contextuals = WordFinal,	
]
\setsansfont{TeX Gyre Adventor}[
Numbers	=	{OldStyle, Proportional},
Ligatures	=	TeX,
Scale=MatchLowercase]
\setmonofont{Inconsolata}[
Scale = MatchLowercase, 
Ligatures = TeX,
]
\setmathfont{XITS Math}
\setmathfont[range=it->up]{Neo Euler}
\setmathfont[range=up/{num}]{Neo Euler}
\newfontface{\andm}{Andale Mono}
\newfontface{\babm}{BabelStone Mayan Numerals}%{Symbola}
\newfontface{\mayanumerals}{BabelStone Mayan Numerals}
\newfontface{\charis}{Charis SIL}%{Gentium Plus}
\newfontface{\charisbf}{Charis SIL Bold}%{Charis SIL}
%\newfontface{\palarab}{PalatinoLTArabic}%{Gentium Plus}
%%%%%%%%%%%%
\linespread{1.05}% Palatino needs more leading (space between lines) {1.01} {1.08}
%\usepackage{ucs}
\usepackage[spanish,mexico]{babel}
%\usepackage{amsmath}
%\usepackage{amsfonts}
%\usepackage{amssymb}
%\usepackage{makeidx}
\usepackage[format=hang,font=small,labelfont=bf,labelsep=quad]{caption}
\usepackage{xcolor}
\usepackage{graphicx}
\usepackage[useregional]{datetime2}
\usepackage{fancyhdr}
\usepackage{tikz}
\usetikzlibrary{arrows}
\usepackage[colorlinks=true,urlcolor=brown]{hyperref}
% % % %Geometry
\usepackage{geometry}%\usepackage[showframe]{geometry}
%\usepackage{layout}
\setlength{\voffset}{-0.7in}
\setlength{\headsep}{10pt}
\setlength{\textheight}{10.5in}
%\usepackage{coloremoji}
\usepackage{wasysym} %emoticons :)
%\usepackage[oldstyle]{kpfonts}
%\usepackage[T1]{fontenc}^^^^1dc4
\newcommand{\fej}{\relax\hfill\ifmmode{\lower.5ex\hbox{{\textcolor{blue}{\LARGE\smiley al 15pt}}}}\else\lower.5ex\hbox{{\textcolor{blue}{\LARGE \smiley}}}}  % Smiley emoticon :)
%\newcommand{\fej}{\relax\hfill\ifmmode{\lower.5ex\hbox{{\textcolor{blue}{\LARGE\smiley}}}}\else\lower.5ex\hbox{{\textcolor{blue}{\LARGE \smiley}}}}  % Smiley emoticon :)
\author{\textsc{Manuel López Mateos}}
%%%% Diciembre 17,2019
% % % % % % % Para usar título, autor y fecha por separado.
\makeatletter
\let\newtitle\@title
\let\elautor\@author
\let\newdate\@date
\makeatother
%
%
% % % % Enviroments
\newenvironment{definition}[1][Definición.]{\begin{trivlist}
\item[\hskip \labelsep {\bfseries #1}]}{\end{trivlist}}
% % % % % % % % % % % %
%
% % % % % % % % Headers
\pagestyle{fancy}
\fancyhf{}
\rhead{\color{olive}\hfill \DTMnow}
\lhead{\color{olive}\elautor}
\cfoot{\thepage}
\renewcommand{\headrule}{\color{olive}\hrule}
\newcommand{\R}{\relax\ifmmode\mathbb{R}\else${\mathbb{R}}$\fi}
\newcommand*{\curn}[1]{{\bfseries\itshape#1}}
\newcommand{\nin}{\noindent}
\newcommand{\LuaLaTeX}{L\kern-0.25em\raise0.5ex\hbox{\tiny U}\kern-0.04em\raise0.5ex\hbox{\tiny A}\kern-0.05em\LaTeX}
\newcommand*{\nota}[2]{{\sffamily\scshape\color{#1}#2.}\enspace}
%\rfoot{}
% % % % % % %
\begin{document} %\layout
%\noindent{\color{purple} \elautor \hfill \DTMnow
%\smallskip
%
%\hrule}
\bigskip 
\nin {\sffamily\scshape\color{orange}Completar Cuadrado}

\nin Expresa $x^2+5x$ como el \emph{cuadrado de un binomio}.

La forma del cuadrado de un binomio es
$$(a+b)^2=a^2+2ab+b^2.$$

Si tratamos de ajustar los términos de la expresión dada en la forma del cuadrado del binomio, tendremos que $a^2=x^2$ y que $5x=2ab$. Vemos de inmediato, que $x=a$. Así, en la forma del cuadrado del binomio tenemos
$$(\underbrace{x}_{\color{blue}a}+{\color{blue}b})^2=\underbrace{x^2}_{\color{blue}a^2}+\underbrace{5x}_{\color{blue}2ab}+{\color{blue}b^2}.$$

Para \emph{completar} el cuadrado en la expresión anterior, falta averiguar quién es $b$. Sabemos que $2ab = 5x$ y que $x=a$, luego $2ab=2xb$ y tenemos entonces, que
$$\underbrace{2xb}_{\color{blue}2ab}=5x.$$

Despejamos $b$ y obtenemos que $b=\frac{5}{2}$.
\smallskip

Substituimos en la forma del cuadrado del binomio,
$$\Big(\underbrace{x}_{\color{blue}a}+\underbrace{\dfrac{5}{2}}_{\color{blue}b}\Big)^2=\underbrace{x^2}_{\color{blue}a^2}+\underbrace{5x}_{\color{blue}2ab}+\underbrace{\Big(\frac{5}{2}\Big)^2}_{\color{blue}b^2}.$$

Ahora despejamos la expresión original,
$$x^2+5x=\Big({x}+\frac{5}{2}\Big)^2-\Big(\frac{5}{2}\Big)^2.$$

Naturalmente, como $x^2 +5x$ no es \emph{el cuadrado} de un binomio, tiene que ajustarse.

Es como si pidiera expresar el número entero $19$ como el cuadrado de un entero. Al no ser un cuadrado, para representarlo como tal, ajustamos:
$$19=4^2+3.$$

\nin\textbf{Otro ejemplo}. Representa \textsc{$3x^2+5x+7$} como el cuadrado de un binomio.

Comenzamos \emph{sacando} al $3$ como factor común, $3x^2+5x+7= 3(x^2+\frac{5}{3}x+\frac{7}{3})$.

Aplicamos el procedimiento a $x^2+\frac{5}{3}x$. En este caso, $2ab=\frac{5}{3}x$, $a=x$ y $b=\frac{5}{6}$. Obtenemos
$$x^2+\frac{5}{3}x={\left(x+\frac{5}{6}\right)}^2-\Big(\frac{5}{6}\Big)^2.$$
Luego
$$x^2+\frac{5}{3}x+\frac{7}{3}={\left(x+\frac{5}{6}\right)}^2-\Big(\frac{5}{6}\Big)^2+\frac{7}{3},$$
y
$$3x^2+5x+7=3\left[{\left(x+\frac{5}{6}\right)}^2-\Big(\frac{5}{6}\Big)^2+\frac{7}{3}\right].$$

Hay que reducir $-\left(\frac{5}{6}\right)^2+\frac{7}{3}=\frac{7}{3} -\left(\frac{5}{6}\right)^2$, subtituirlo y multiplicar por $3$, pero faltó espacio!!!




\fej






\smallskip






\vfill 



\begin{center}
	{\footnotesize\color{olive} Esta hoja se formó con el sistema \LuaLaTeX.}
\end{center}
\end{document}