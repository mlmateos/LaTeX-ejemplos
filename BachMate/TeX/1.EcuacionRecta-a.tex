\documentclass[12pt,letterpaper]{article}
%%% LuaLaTex
\usepackage{fontspec}
\usepackage{amsmath}% if desired
\usepackage{unicode-math}
\renewcommand{\boldsymbol}{\symbf}

\setmainfont{TeX Gyre Pagella}[
Numbers	=	{OldStyle, Proportional},
Ligatures	=	TeX,
%Script=Arabic
%Contextuals = WordFinal,	
]
\setsansfont{TeX Gyre Adventor}[
Numbers	=	{OldStyle, Proportional},
Ligatures	=	TeX,
Scale=MatchLowercase]
\setmonofont{Inconsolata}[
Scale = MatchLowercase, 
Ligatures = TeX,
]
\setmathfont{Asana Math}
\setmathfont[range=it->up]{Neo Euler}
\setmathfont[range=up/{num}]{Neo Euler}
\newfontface{\andm}{Andale Mono}
\newfontface{\babm}{BabelStone Mayan Numerals}%{Symbola}
\newfontface{\mayanumerals}{BabelStone Mayan Numerals}
\newfontface{\charis}{Charis SIL}%{Gentium Plus}
\newfontface{\charisbf}{Charis SIL Bold}%{Charis SIL}
%\newfontface{\palarab}{PalatinoLTArabic}%{Gentium Plus}
%%%%%%%%%%%%
\linespread{1.05}% Palatino needs more leading (space between lines) {1.01} {1.08}
%\usepackage{ucs}
\usepackage[spanish,mexico]{babel}
%\usepackage{amsmath}
%\usepackage{amsfonts}
%\usepackage{amssymb}
%\usepackage{makeidx}
\usepackage{xcolor}
\usepackage{graphicx}
\usepackage[useregional]{datetime2}
\usepackage{fancyhdr}
\usepackage{tikz}
% % % %Geometry
\usepackage{geometry}%\usepackage[showframe]{geometry}
%\usepackage{layout}
\setlength{\voffset}{-0.7in}
\setlength{\headsep}{10pt}
\setlength{\textheight}{10.5in}
\usepackage{wasysym} %emoticons :)
%\usepackage[oldstyle]{kpfonts}
%\usepackage[T1]{fontenc}
\newcommand{\fej}{\relax\hfill\ifmmode{\lower.5ex\hbox{{\textcolor{blue}{\LARGE\smiley al 15pt}}}}\else\lower.5ex\hbox{{\textcolor{blue}{\LARGE \smiley}}}}  % Smiley emoticon :)
\author{\textsc{Manuel López Mateos}}
%%% 31 de octubre de 2019
% % % % % % % Para usar título, autor y fecha por separado.
\makeatletter
\let\newtitle\@title
\let\elautor\@author
\let\newdate\@date
\makeatother
%
%
% % % % Enviroments
\newenvironment{definition}[1][Definición.]{\begin{trivlist}
\item[\hskip \labelsep {\bfseries #1}]}{\end{trivlist}}
% % % % % % % % % % % %
%
% % % % % % % % Headers
\pagestyle{fancy}
\fancyhf{}
\rhead{\color{olive}\hfill \DTMnow}
\lhead{\color{olive}\elautor}
\cfoot{\thepage}
\renewcommand{\headrule}{\color{olive}\hrule}
\newcommand{\R}{\relax\ifmmode\mathbb{R}\else${\mathbb{R}}$\fi}
%\rfoot{}
% % % % % % %
\begin{document} %\layout
%\noindent{\color{purple} \elautor \hfill \DTMnow
%\smallskip
%
%\hrule}
\bigskip 

\noindent La \emph{\color{purple}forma canónica} de la ecuación de una recta en el plano $\R^2$, es 
$$y=mx+b,$$
donde $m$ es la \emph{\color{purple}pendiente} y $b$ es la \emph{\color{purple}ordenada al origen}; $x$ es la \emph{\color{purple}variable independiente}, $y$ es la \emph{\color{purple}variable dependiente}: a cada valor de $x$ corresponde un valor de $y$.

En el lenguaje de \textbf{\color{purple}conjuntos}, decimos que la recta $L$ es el \emph{conjunto de puntos $(x,y)$ del plano} $\R^2$, tales que satisfacen la ecuación $y=mx+b$. Se escribe
$$L=\left\{\,(x,y)\in\R^2\mid y=mx+b\,\right\}.$$

\begin{figure}[ht]
	\begin{center}
\begin{tikzpicture}[scale=1]
\draw[style=help lines] (-3.9,-3.9) grid (3.9,3.9);

\draw[->] (-3.2,0) -- (4,0) node[right] {$x$};
\draw[->] (0,-3.2) -- (0,4) node[above] {$y$};
\draw[thick,blue] (-2.75,-3.5) -- (1,4);

\foreach \x/\xtext in {-3/-3, -2/-2, -1/-1, 0/0, 1/1, 2/2, 3/3}
\draw[shift={(\x,0)}] (0pt,2pt) -- (0pt,-2pt) node[below left] {{\small$\xtext$}};

\foreach \y/\ytext in {-3/-3, -2/-2, -1/-1, 1/1, 2/2, 3/3}
\draw[shift={(0,\y)}] (2pt,0pt) -- (-2pt,0pt) node[below left] {\small$\ytext$};
\end{tikzpicture}
\end{center}
\caption{En azul la gráfica de $y=2x+2$.}\label{fig:reccan}
\end{figure}

Para hallar los \emph{\color{purple}cruces} de la recta de la Figura~\ref{fig:reccan} con los ejes, en la ecuación $y=2x+2$ hacemos primero $x=0$ para obtener el cruce con el eje $y$, obtenemos que $y=2$, la ordenada al origen; es decir, el cruce de la recta $y=2x+2$ con el eje $y$ es el punto $(0,2)$; y después hacemos $y=0$ para obtener el cruce con el eje $x$, obtenemos que $x=-1$, así, el cruce con el eje $x$ de la recta de la figura, es el punto $(-1,0)$.

A la abscisa $x$ del punto de cruce con el eje $x$ se le llama la \emph{\color{purple}raíz} de la ecuación.

La \emph{\color{purple}pendiente} es una \textbf{\color{purple}razón} o \textbf{\color{purple}cociente} que \emph{\color{purple}compara} lo que ``sube'' la recta contra lo que ``avanza''. Dicho propiamente, compara el \emph{\color{purple}incremento de las ordenadas} contra el \emph{\color{purple}incremento de las abscisas}. Si denotamos por $\Delta y$ el incremento de las ordenadas y $\Delta x$ el de las abscisas, la pendiente de la recta es igual a
$$m=\frac{\Delta y}{\Delta x}.$$

Si una recta pasa por los puntos $A=(x_1, y_1)$ y $B=(x_2, y_2)$ su pendiente, que denotamos por $m_{AB}$ es igual a
$$m_{AB}=\frac{\Delta y}{\Delta x}=\frac{y_2-y_1}{x_2-x_1}.$$

\fej
\end{document}