% !TeX TS-program = lualatex
\documentclass[12pt,letterpaper]{article}
%%% LuaLaTex
\usepackage{fontspec}
\usepackage{amsmath}% if desired
\usepackage{unicode-math}
\renewcommand{\boldsymbol}{\symbf}

\setmainfont{TeX Gyre Pagella}[
Numbers	=	{OldStyle, Proportional},
Ligatures	=	TeX,
%Script=Arabic
%Contextuals = WordFinal,	
]
\setsansfont{TeX Gyre Adventor}[
Numbers	=	{OldStyle, Proportional},
Ligatures	=	TeX,
Scale=MatchLowercase]
\setmonofont{Inconsolata}[
Scale = MatchLowercase, 
Ligatures = TeX,
]
\setmathfont{XITS Math}
\setmathfont[range=it->up]{Neo Euler}
\setmathfont[range=up/{num}]{Neo Euler}
\newfontface{\andm}{Andale Mono}
\newfontface{\babm}{BabelStone Mayan Numerals}%{Symbola}
\newfontface{\mayanumerals}{BabelStone Mayan Numerals}
\newfontface{\charis}{Charis SIL}%{Gentium Plus}
\newfontface{\charisbf}{Charis SIL Bold}%{Charis SIL}
%\newfontface{\palarab}{PalatinoLTArabic}%{Gentium Plus}
%%%%%%%%%%%%
\linespread{1.05}% Palatino needs more leading (space between lines) {1.01} {1.08}
%\usepackage{ucs}
\usepackage[spanish,mexico]{babel}
%\usepackage{amsmath}
%\usepackage{amsfonts}
%\usepackage{amssymb}
%\usepackage{makeidx}
\usepackage[format=hang,font=small,labelfont=bf,labelsep=quad]{caption}
\usepackage{xcolor}
\usepackage{graphicx}
\usepackage[useregional]{datetime2}
\usepackage{fancyhdr}
\usepackage{tikz}
\usetikzlibrary{arrows}
\usepackage[colorlinks=true,urlcolor=brown]{hyperref}
% % % %Geometry
\usepackage{geometry}%\usepackage[showframe]{geometry}
%\usepackage{layout}
\setlength{\voffset}{-0.7in}
\setlength{\headsep}{10pt}
\setlength{\textheight}{10.5in}
%\usepackage{coloremoji}
\usepackage{wasysym} %emoticons :)
%\usepackage[oldstyle]{kpfonts}
%\usepackage[T1]{fontenc}^^^^1dc4
\newcommand{\fej}{\relax\hfill\ifmmode{\lower.5ex\hbox{{\textcolor{blue}{\LARGE\smiley al 15pt}}}}\else\lower.5ex\hbox{{\textcolor{blue}{\LARGE \smiley}}}}  % Smiley emoticon :)
%\newcommand{\fej}{\relax\hfill\ifmmode{\lower.5ex\hbox{{\textcolor{blue}{\LARGE\smiley}}}}\else\lower.5ex\hbox{{\textcolor{blue}{\LARGE \smiley}}}}  % Smiley emoticon :)
\author{\textsc{Manuel López Mateos}}
%%%% Diciembre 12, 2019
% % % % % % % Para usar título, autor y fecha por separado.
\makeatletter
\let\newtitle\@title
\let\elautor\@author
\let\newdate\@date
\makeatother
%
%
% % % % Enviroments
\newenvironment{definition}[1][Definición.]{\begin{trivlist}
\item[\hskip \labelsep {\bfseries #1}]}{\end{trivlist}}
% % % % % % % % % % % %
%
% % % % % % % % Headers
\pagestyle{fancy}
\fancyhf{}
\rhead{\color{olive}\hfill \DTMnow}
\lhead{\color{olive}\elautor}
\cfoot{\thepage}
\renewcommand{\headrule}{\color{olive}\hrule}
\newcommand{\R}{\relax\ifmmode\mathbb{R}\else${\mathbb{R}}$\fi}
\newcommand*{\curn}[1]{{\bfseries\itshape#1}}
\newcommand{\nin}{\noindent}
\newcommand{\LuaLaTeX}{L\kern-0.25em\raise0.5ex\hbox{\tiny U}\kern-0.04em\raise0.5ex\hbox{\tiny A}\kern-0.05em\LaTeX}
\newcommand*{\nota}[2]{{\sffamily\scshape\color{#1}#2.}\enspace}
%\rfoot{}
% % % % % % %
\begin{document} %\layout
%\noindent{\color{purple} \elautor \hfill \DTMnow
%\smallskip
%
%\hrule}
\bigskip 
\nin {\sffamily\scshape\color{orange}Repaso necesario.}\enspace 
\begin{itemize}
	\item\nota{gray}{Factor común} $(ax+ay)=a(x+y)$. Decimos, en la igualdad anterior, que \emph{sacamos} $a$ como \emph{factor común}. En realidad estamos empleando la \curn{ley distributiva} de los números reales: $a(b+c)=ab+ac$, lo cual, a su vez, es igual~a\break $(b+c)a$, debido a la \curn{ley conmutativa} (\emph{el orden de los factores no altera el producto}). Así, tenemos que $ax+a=a(x+1)$, o que $3x+6=3(x+2)$.
	\item\nota{gray}{Producto de dos binomios}Para multiplicar $(a+b)(c+d)$ tomamos un factor, digamos $(a+b)$ como un término, y aplicamos la ley distributiva
\begin{align*}
	 {\color{blue}(a+b)}{\color{brown}(c+d)}&={\color{blue}(a+b)}{\color{brown}c}+{\color{blue}(a+b)}{\color{brown}d}\\
	 &={\color{blue}a}{\color{brown}c}+{\color{blue}b}{\color{brown}c}+{\color{blue}a}{\color{brown}d}+{\color{blue}b}{\color{brown}d}.
\end{align*}
\begin{figure}[h]\centering
\begin{tikzpicture}[scale=1]
% 1. Lado horizontal de longitud a+b
\draw [very thick, blue] (0,0)  -- (2,0)  node[midway,below] {\small$\color{blue}a$};
\draw [very thick, blue] (2,0)  -- (3,0)  node[midway,below] {\small$\color{blue}b$};
% 2. Lado horizontal de longitud a+b
\draw [dashed, blue] (0,1.5)  -- (2,1.5)  node[midway,below] {\small$\color{blue}a$};
\draw [dashed, blue] (2,1.5)  -- (3,1.5)  node[midway,below] {\small$\color{blue}b$};
% 3. Lado horizontal de longitud a+b
\draw [dashed, blue] (0,4)  -- (2,4)  node[midway,below] {\small$\color{blue}a$};
\draw [dashed, blue] (2,4)  -- (3,4)  node[midway,below] {\small$\color{blue}b$};
% 1.Lado vertical de longitud c+d
\draw [very thick, brown] (3,0)  -- (3,1.5)  node[midway,right] {\small$\color{brown}c$};
\draw [very thick, brown] (3,1.5)  -- (3,4)  node[midway,right] {\small$\color{brown}d$};
% 2.Lado vertical de longitud c+d
\draw [dashed, brown] (2,0)  -- (2,1.5)  node[midway,right] {\small$\color{brown}c$};
\draw [dashed, brown] (2,1.5)  -- (2,4)  node[midway,right] {\small$\color{brown}d$};
% 3.Lado vertical de longitud c+d
\draw [dashed, brown] (0,0)  -- (0,1.5)  node[midway,right] {\small$\color{brown}c$};
\draw [dashed, brown] (0,1.5)  -- (0,4)  node[midway,right] {\small$\color{brown}d$};
%Marcas en el lado horizontal
\draw [blue] (0,0) node {$|$} ;
\draw [blue] (2,0) node {$|$} ;
\draw [blue] (3,0) node {$|$} ;
%Marcas en el aldo vertical
\draw [brown] (3,-0.05) node {$-$} ;
\draw [brown] (3,1.5-0.05) node {$-$} ;
\draw [brown] (3,4-0.05) node {$-$} ;
\end{tikzpicture}
	\caption{El \emph{área} del rectángulo de lados $(a+b)$ y $(c+d)$ es la \emph{suma} de las áreas $ac$, $bc$, $ad$ y $bd$.}
\end{figure}
	\item \nota{gray}{Cuadrado de un binomio}Por el punto anterior, tenemos que
	\begin{align*}
	(a+b)^2&=(a+b)(a+b)\\
	&=(a+b)a+(a+b)b\\
	&=aa+ba+ab+bb\\
	&=a^2+2ab+b^2.
	\end{align*}
	\begin{figure}[h]\centering
		\begin{tikzpicture}[scale=1]
		% 1. Lado horizontal de longitud a+b
		\draw [very thick, blue] (0,0)  -- (2,0)  node[midway,below] {\small$\color{blue}a$};
		\draw [very thick, blue] (2,0)  -- (3,0)  node[midway,below] {\small$\color{blue}b$};
		% 2. Lado horizontal de longitud a+b
		\draw [dashed, blue] (0,2)  -- (2,2)  node[midway,below] {\small$\color{blue}a$};
		\draw [dashed, blue] (2,2)  -- (3,2)  node[midway,below] {\small$\color{blue}b$};
		% 3. Lado horizontal de longitud a+b
		\draw [dashed, blue] (0,3)  -- (2,3)  node[midway,below] {\small$\color{blue}a$};
		\draw [dashed, blue] (2,3)  -- (3,3)  node[midway,below] {\small$\color{blue}b$};
		% 1.Lado vertical de longitud c+d
		\draw [very thick, brown] (3,0)  -- (3,2)  node[midway,right] {\small$\color{brown}a$};
		\draw [very thick, brown] (3,2)  -- (3,3)  node[midway,right] {\small$\color{brown}b$};
		% 2.Lado vertical de longitud c+d
		\draw [dashed, brown] (2,0)  -- (2,2)  node[midway,right] {\small$\color{brown}a$};
		\draw [dashed, brown] (2,2)  -- (2,3)  node[midway,right] {\small$\color{brown}b$};
		% 3.Lado vertical de longitud c+d
		\draw [dashed, brown] (0,0)  -- (0,2)  node[midway,right] {\small$\color{brown}a$};
		\draw [dashed, brown] (0,2)  -- (0,3)  node[midway,right] {\small$\color{brown}b$};
		%Marcas en el lado horizontal
		\draw [blue] (0,0) node {$|$} ;
		\draw [blue] (2,0) node {$|$} ;
		\draw [blue] (3,0) node {$|$} ;
		%Marcas en el aldo vertical
		\draw [brown] (3,-0.05) node {$-$} ;
		\draw [brown] (3,2-0.05) node {$-$} ;
		\draw [brown] (3,3-0.05) node {$-$} ;
		\end{tikzpicture}
		\caption{El \emph{área} del \curn{cuadrado} de lado $(a+b)$ es la \emph{suma} de las áreas $aa$, $ba$, $ab$ y $ba$.}
	\end{figure}
	
\end{itemize}

\fej






\smallskip






\vfill 



\begin{center}
	{\footnotesize\color{olive} Esta hoja se formó con el sistema \LuaLaTeX.}
\end{center}
\end{document}